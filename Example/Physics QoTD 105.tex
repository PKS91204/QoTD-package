\begin{qotdphybox}{DRUNKEN ATOM}
\qotdtitle{105}{13}{05}
The potential energy of an atom in some crystal is described by the formula
$$\fbox{$U (r)  = U_\circ \qty(\qty(\frac{r_\circ}{r})^{12} - 2 \qty(\frac{r}{r_\circ})^6)$}\ ,$$
where $U_\circ = 8.8\times 10^{-4} ~ \mathrm{eV}$ and $r_\circ = 0.287 ~ \mathrm{nm}$ corresponds to the equilibrium position of the atom. 

For small deviation from the equilibrium position, oscillation occur.\\
According to quantum concepts, the energy of oscillation with a frequency $\omega = 2\pi \nu$ can take values $E_n = h \nu \qty(n + \frac{1}{2}),~n = 0,1,2,\dots$, where $h = 6.62\times 10^{-34} ~ \mathrm{J\cdot s}$ is Planck's constant.
\begin{qotdques}
\ques{12} Estimate the smallest amplitude $\chi_\circ$ oscillations of the displacement of an atom in such a crystal.
\end{qotdques}
\textcolor{teal}{[Mass of an atom $m = 6.4 \times 10^{-24}~ \mathrm{g}$; $1 ~ \mathrm{eV} = 1.6 \times 10^{-19} ~ \mathrm{J}$]}\\
\qotdcreator{JØKÊR}
\end{qotdphybox}